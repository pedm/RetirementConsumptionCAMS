\begin{table}[tbp] \centering
\newcolumntype{C}{>{\centering\arraybackslash}X}

\caption{Real household income before and after retirement by wealth tertiles.}
\begin{tabularx}{\textwidth}{lCCCC}

\toprule
{Wealth\_Tertiles}&{First}&{Second}&{Third}&{All} \tabularnewline
\midrule\addlinespace[1.5ex]
Means:&&&& \tabularnewline
\midrule Pre-retirement&32,619&58,499&115,190&68,208 \tabularnewline
Post-retirement&28,215&56,089&116,776&67,573 \tabularnewline
Percent Change in Means&-13.5&-7.4&4.9&-0.9 \tabularnewline
\midrule Medians:&&&& \tabularnewline
\midrule Pre-retirement&23,370&52,263&87,740&48,920 \tabularnewline
Post-retirement&20,200&50,412&80,520&46,512 \tabularnewline
Percent Change in Medians&-13.6&-3.5&-8.2&-4.9 \tabularnewline
Median Percent Change (p10)*&-0.7&-0.5&-0.6&-0.6 \tabularnewline
Median Percent Change (p25)*&-0.4&-0.2&-0.3&-0.3 \tabularnewline
Median Percent Change (p50)&-0.1&0.0&0.0&0.0 \tabularnewline
Median Percent Change (p75)*&0.3&0.2&0.3&0.3 \tabularnewline
Median Percent Change (p90)*&1.1&0.7&1.1&1.0 \tabularnewline
\bottomrule \addlinespace[1.5ex]

\end{tabularx}
\begin{flushleft}
\footnotesize *These values are not medians but percentiles, as indicated in the parentheses. \linebreak --- \linebreak This table references Table 3 of Hurd and Rohwedder's paper: Heterogeneity in spending change at retirement. \linebreak --- \linebreak Mean percent change is not reported because observation error on spending can produce large outliers when spending is put in ratio form. \linebreak --- \linebreak N = 919.
\end{flushleft}
\end{table}
