\begin{table}[tbp] \centering
\newcolumntype{C}{>{\centering\arraybackslash}X}

\caption{Real food away from home spending before and after retirement by wealth tertiles (PSID category).}
\begin{tabularx}{\textwidth}{lCCCC}

\toprule
{Wealth\_Tertiles}&{First}&{Second}&{Third}&{All} \tabularnewline
\midrule\addlinespace[1.5ex]
Means:&&&& \tabularnewline
\midrule Pre-retirement&897&1,654&2,246&1,597 \tabularnewline
Post-retirement&799&1,632&2,391&1,624 \tabularnewline
Percent Change in Means&-11.0&-2.5&16.2&1.7 \tabularnewline
\midrule Medians:&&&& \tabularnewline
\midrule Pre-retirement&393&1,043&1,706&983 \tabularnewline
Post-retirement&360&932&1,608&932 \tabularnewline
Percent Change in Medians&-8.3&-10.7&-5.7&-5.2 \tabularnewline
Median Percent Change (p10)*&-1.0&-0.8&-0.7&-0.9 \tabularnewline
Median Percent Change (p25)*&-0.8&-0.5&-0.4&-0.5 \tabularnewline
Median Percent Change (p50)&-0.2&-0.1&0.0&-0.1 \tabularnewline
Median Percent Change (p75)*&0.9&0.6&0.6&0.7 \tabularnewline
Median Percent Change (p90)*&3.2&3.5&1.9&2.6 \tabularnewline
\bottomrule \addlinespace[1.5ex]

\end{tabularx}
\begin{flushleft}
\footnotesize *These values are not medians but percentiles, as indicated in the parentheses. \linebreak --- \linebreak This table references Table 2 of Hurd and Rohwedder's paper: Heterogeneity in spending change at retirement. \linebreak --- \linebreak This spending category is defined by dining out in CAMS. \linebreak --- \linebreak Mean percent change is not reported because observation error on spending can produce large outliers when spending is put in ratio form. \linebreak --- \linebreak N = 893.
\end{flushleft}
\end{table}
