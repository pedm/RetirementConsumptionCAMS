\begin{table}[tbp] \centering
\newcolumntype{C}{>{\centering\arraybackslash}X}

\caption{Real housing spending before and after retirement by social security income tertiles (CAMS and PSID category).}
\begin{tabularx}{\textwidth}{lCCCC}

\toprule
{SS\_Tertiles}&{First}&{Second}&{Third}&{All} \tabularnewline
\midrule\addlinespace[1.5ex]
Means:&&&& \tabularnewline
\midrule Pre-retirement&6,425&8,536&13,354&10,420 \tabularnewline
Post-retirement&6,085&8,076&13,133&10,072 \tabularnewline
Percent Change in Means&-5.3&-7.2&-3.4&-3.3 \tabularnewline
\midrule Medians:&&&& \tabularnewline
\midrule Pre-retirement&3,466&5,661&9,555&7,219 \tabularnewline
Post-retirement&3,804&5,674&9,220&6,493 \tabularnewline
Percent Change in Medians&9.8&0.2&-3.5&-10.1 \tabularnewline
Median Percent Change (p10)*&-0.8&-0.7&-0.7&-0.7 \tabularnewline
Median Percent Change (p25)*&-0.4&-0.4&-0.4&-0.4 \tabularnewline
Median Percent Change (p50)&0.0&0.0&0.0&0.0 \tabularnewline
Median Percent Change (p75)*&0.8&0.3&0.3&0.3 \tabularnewline
Median Percent Change (p90)*&4.3&1.7&2.1&2.1 \tabularnewline
\bottomrule \addlinespace[1.5ex]

\end{tabularx}
\begin{flushleft}
\footnotesize *These values are not medians but percentiles, as indicated in the parentheses. \linebreak --- \linebreak This table references Table 2 of Hurd and Rohwedder's paper: Heterogeneity in spending change at retirement. \linebreak --- \linebreak This spending category is defined in accordance with page 9 (Table 1: Variable Names Across Waves) of the RAND\_CAMS\_2015V2 Data Documentation file. \linebreak --- \linebreak Mean percent change is not reported because observation error on spending can produce large outliers when spending is put in ratio form. \linebreak --- \linebreak N = 919.
\end{flushleft}
\end{table}
