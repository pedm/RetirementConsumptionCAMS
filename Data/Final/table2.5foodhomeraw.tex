\begin{table}[tbp] \centering
\newcolumntype{C}{>{\centering\arraybackslash}X}

\caption{Real food at home spending before and after retirement by social security income tertiles (PSID category).}
\begin{tabularx}{\textwidth}{lCCCC}

\toprule
{SS\_Tertiles}&{First}&{Second}&{Third}&{All} \tabularnewline
\midrule\addlinespace[1.5ex]
Means:&&&& \tabularnewline
\midrule Pre-retirement&3,888&3,595&4,362&3,952 \tabularnewline
Post-retirement&3,655&3,776&4,261&3,977 \tabularnewline
Percent Change in Means&-6.0&4.7&-2.6&0.6 \tabularnewline
\midrule Medians:&&&& \tabularnewline
\midrule Pre-retirement&3,614&2,986&3,793&3,354 \tabularnewline
Post-retirement&2,607&2,972&3,600&3,195 \tabularnewline
Percent Change in Medians&-27.9&-0.5&-5.1&-4.7 \tabularnewline
Median Percent Change (p10)*&-0.7&-0.7&-0.7&-0.7 \tabularnewline
Median Percent Change (p25)*&-0.5&-0.4&-0.3&-0.4 \tabularnewline
Median Percent Change (p50)&-0.1&0.0&0.0&0.0 \tabularnewline
Median Percent Change (p75)*&0.4&0.6&0.5&0.5 \tabularnewline
Median Percent Change (p90)*&2.3&1.7&1.6&1.7 \tabularnewline
\bottomrule \addlinespace[1.5ex]

\end{tabularx}
\begin{flushleft}
\footnotesize *These values are not medians but percentiles, as indicated in the parentheses. \linebreak --- \linebreak This table references Table 2 of Hurd and Rohwedder's paper: Heterogeneity in spending change at retirement. \linebreak --- \linebreak This spending category is defined by food/drink in CAMS. \linebreak --- \linebreak Mean percent change is not reported because observation error on spending can produce large outliers when spending is put in ratio form. \linebreak --- \linebreak N = 896.
\end{flushleft}
\end{table}
