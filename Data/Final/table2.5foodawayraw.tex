\begin{table}[tbp] \centering
\newcolumntype{C}{>{\centering\arraybackslash}X}

\caption{Real food away from home spending before and after retirement by social security income tertiles (PSID category).}
\begin{tabularx}{\textwidth}{lCCCC}

\toprule
{SS\_Tertiles}&{First}&{Second}&{Third}&{All} \tabularnewline
\midrule\addlinespace[1.5ex]
Means:&&&& \tabularnewline
\midrule Pre-retirement&962&1,354&1,991&1,597 \tabularnewline
Post-retirement&975&1,309&2,100&1,624 \tabularnewline
Percent Change in Means&1.4&-4.6&11.4&1.7 \tabularnewline
\midrule Medians:&&&& \tabularnewline
\midrule Pre-retirement&416&664&1,315&983 \tabularnewline
Post-retirement&474&678&1,286&932 \tabularnewline
Percent Change in Medians&14.0&2.2&-2.2&-5.2 \tabularnewline
Median Percent Change (p10)*&-100.0&-89.2&-78.4&-87.2 \tabularnewline
Median Percent Change (p25)*&-78.3&-56.6&-51.7&-52.9 \tabularnewline
Median Percent Change (p50)&-12.0&-5.8&-4.6&-5.8 \tabularnewline
Median Percent Change (p75)*&92.4&62.6&67.8&65.8 \tabularnewline
Median Percent Change (p90)*&318.1&238.9&268.6&262.1 \tabularnewline
\bottomrule \addlinespace[1.5ex]

\end{tabularx}
\begin{flushleft}
\footnotesize *These values are not medians but percentiles, as indicated in the parentheses. \linebreak --- \linebreak This table references Table 2 of Hurd and Rohwedder's paper: Heterogeneity in spending change at retirement. \linebreak --- \linebreak This spending category is defined by dining out in CAMS. \linebreak --- \linebreak Mean percent change is not reported because observation error on spending can produce large outliers when spending is put in ratio form. \linebreak --- \linebreak N = 893.
\end{flushleft}
\end{table}
