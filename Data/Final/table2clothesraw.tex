\begin{table}[tbp] \centering
\newcolumntype{C}{>{\centering\arraybackslash}X}

\caption{Real clothing spending before and after retirement by wealth tertiles (PSID category).}
\begin{tabularx}{\textwidth}{lCCCC}

\toprule
{Wealth\_Tertiles}&{First}&{Second}&{Third}&{All} \tabularnewline
\midrule\addlinespace[1.5ex]
Means:&&&& \tabularnewline
\midrule Pre-retirement&756&1,021&1,165&979 \tabularnewline
Post-retirement&537&736&1,326&872 \tabularnewline
Percent Change in Means&-29.0&-37.6&21.3&-10.9 \tabularnewline
\midrule Medians:&&&& \tabularnewline
\midrule Pre-retirement&266&474&624&471 \tabularnewline
Post-retirement&206&388&565&392 \tabularnewline
Percent Change in Medians&-22.7&-18.1&-9.3&-16.9 \tabularnewline
Median Percent Change (p10)*&-1.0&-0.9&-0.8&-0.9 \tabularnewline
Median Percent Change (p25)*&-0.8&-0.7&-0.6&-0.7 \tabularnewline
Median Percent Change (p50)&-0.2&-0.2&-0.2&-0.2 \tabularnewline
Median Percent Change (p75)*&0.9&0.5&0.4&0.5 \tabularnewline
Median Percent Change (p90)*&2.3&2.1&1.6&1.9 \tabularnewline
\bottomrule \addlinespace[1.5ex]

\end{tabularx}
\begin{flushleft}
\footnotesize *These values are not medians but percentiles, as indicated in the parentheses. \linebreak --- \linebreak This table references Table 2 of Hurd and Rohwedder's paper: Heterogeneity in spending change at retirement. \linebreak --- \linebreak This spending category is defined by clothing in CAMS. \linebreak --- \linebreak Mean percent change is not reported because observation error on spending can produce large outliers when spending is put in ratio form. \linebreak --- \linebreak N = 882.
\end{flushleft}
\end{table}
