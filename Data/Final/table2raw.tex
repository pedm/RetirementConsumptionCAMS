\begin{table}[tbp] \centering
\newcolumntype{C}{>{\centering\arraybackslash}X}

\caption{Real nondurable spending before and after retirement.}
\begin{tabularx}{\textwidth}{lCCCC}

\toprule
{Wealth\_Quartiles}&{First}&{Second}&{Third}&{All} \tabularnewline
\midrule\addlinespace[1.5ex]
Means:&&&& \tabularnewline
\midrule Pre-retirement&28,194&41,050&45,457&37,876 \tabularnewline
Post-retirement&25,688&42,330&51,000&40,124 \tabularnewline
Percent Change in Means&-8.9&4.5&19.7&5.9 \tabularnewline
\midrule Medians:&&&& \tabularnewline
\midrule Pre-retirement&21,510&34,089&36,189&31,440 \tabularnewline
Post-retirement&18,558&33,598&41,244&32,039 \tabularnewline
Percent Change in Medians&-13.7&-2.3&23.5&1.9 \tabularnewline
Median Percent Change (p10)*&-0.7&-0.5&-0.4&-0.5 \tabularnewline
Median Percent Change (p25)*&-0.4&-0.2&-0.2&-0.2 \tabularnewline
Median Percent Change (p50)&-0.1&0.0&0.1&0.0 \tabularnewline
Median Percent Change (p75)*&0.4&0.5&0.4&0.4 \tabularnewline
Median Percent Change (p90)*&1.2&1.3&0.8&1.1 \tabularnewline
\bottomrule \addlinespace[1.5ex]

\end{tabularx}
\begin{flushleft}
\footnotesize *These values are not medians but percentiles, as indicated in the parentheses. \linebreak --- \linebreak This table references Table 2 of Hurd and Rohwedder's paper: Heterogeneity in spending change at retirement. \linebreak --- \linebreak Mean percent change is not reported because observation error on spending can produce large outliers when spending is put in ratio form. \linebreak --- \linebreak N = 443.
\end{flushleft}
\end{table}
