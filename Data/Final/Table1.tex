\documentclass{article}
\usepackage{booktabs}
\usepackage{tabularx}
\usepackage[margin=1in]{geometry}
\begin{document}

\begin{table}[tbp] \centering
\newcolumntype{C}{>{\centering\arraybackslash}X}

\caption{Average and median real spending before and after retirement}
\begin{tabularx}{\textwidth}{lCCC}

\toprule
{Spending}&{Total}&{Nondurables}&{Food} \tabularnewline
\midrule\addlinespace[1.5ex]
Means:&&& \tabularnewline
\midrule Pre-retirement&41653.43&37733.64&5954.679 \tabularnewline
Post-retirement&43788.8&39888.31&6076.811 \tabularnewline
Population Percent change&-4.877&-5.402&-2.01 \tabularnewline
95\% confidence interval&&& \tabularnewline
\midrule Medians:&&& \tabularnewline
\midrule Pre-retirement&35049.96&31459&5090.506 \tabularnewline
Post-retirement&34372.61&32058.36&4913.428 \tabularnewline
Population Percent change&1.971&-1.87&3.604 \tabularnewline
95\% confidence interval&&& \tabularnewline
Household-level change&-.783&2.636&-2.675 \tabularnewline
95\% confidence interval&&& \tabularnewline
\bottomrule \addlinespace[1.5ex]

\end{tabularx}
\begin{flushleft}
\footnotesize Population percent change is calculated as the differences of the means (or medians), and household-level change is calculated as the median of the differences. \linebreak --- \linebreak The average change at the household level is not reported because observation error on spending can produce large outliers when spending is put in ratio form. \linebreak --- \linebreak Retirement sample, N = 442.
\end{flushleft}
\end{table}
\end{document}
