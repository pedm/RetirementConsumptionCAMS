\begin{table}[tbp] \centering
\newcolumntype{C}{>{\centering\arraybackslash}X}

\caption{Average and median real spending without retirement transition}
\begin{tabularx}{\textwidth}{lCCC}

\toprule
{Spending}&{Total}&{Nondurables}&{Food} \tabularnewline
\midrule\addlinespace[1.5ex]
Means:&&& \tabularnewline
\midrule Pre-wave&43,759&38,709&5,928 \tabularnewline
Post-wave&42,409&37,967&5,929 \tabularnewline
Percent Change in Means&-3.1&-1.9&0.0 \tabularnewline
95\% confidence interval&.&.&. \tabularnewline
\midrule Medians:&&& \tabularnewline
\midrule Pre-wave&33,713&30,801&4,693 \tabularnewline
Post-wave&32,929&30,034&4,693 \tabularnewline
Percent Change in Medians&-2.3&-2.5&0.0 \tabularnewline
95\% confidence interval&.&.&. \tabularnewline
Median Percent Change (p10)*&-53.0&-48.6&-66.6 \tabularnewline
Median Percent Change (p25)*&-31.3&-25.9&-36.7 \tabularnewline
Median Percent Change (p50)&-2.5&-1.1&-3.8 \tabularnewline
Median Percent Change (p75)*&37.4&30.4&44.3 \tabularnewline
Median Percent Change (p90)*&153.5&86.5&153.5 \tabularnewline
95\% confidence interval (p50)&.&.&. \tabularnewline
\bottomrule \addlinespace[1.5ex]

\end{tabularx}
\begin{flushleft}
\footnotesize *These values are not medians but percentiles, as indicated in the parentheses. \linebreak --- \linebreak This table references Table 1 of Hurd and Rohwedder's paper: Heterogeneity in spending change at retirement. Hurd and Rohwedder bootstrap their confidence intervals. \linebreak --- \linebreak Mean percent change is not reported because observation error on spending can produce large outliers when spending is put in ratio form. \linebreak --- \linebreak Comparison sample, N = 3877. This sample consists of households whose respondents reported no retirement transition between waves (retired to retired, or not retired to not retired). The comparison sample is weighted to match the composition of the retirement sample with respect to age and marital status and wave.
\end{flushleft}
\end{table}
