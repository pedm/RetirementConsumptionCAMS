\begin{table}[tbp] \centering
\newcolumntype{C}{>{\centering\arraybackslash}X}

\caption{Average and median real spending before and after retirement}
\begin{tabularx}{\textwidth}{lCCC}

\toprule
{Spending}&{Total}&{Nondurables}&{Food} \tabularnewline
\midrule\addlinespace[1.5ex]
Means:&&& \tabularnewline
\midrule Pre-retirement&41,812&37,876&5,930 \tabularnewline
Post-retirement&44,044&40,124&6,091 \tabularnewline
Percent Change in Means&5.3&5.9&2.7 \tabularnewline
95\% confidence interval&.&.&. \tabularnewline
\midrule Medians:&&& \tabularnewline
\midrule Pre-retirement&35,009&31,440&4,989 \tabularnewline
Post-retirement&34,376&32,039&4,913 \tabularnewline
Percent Change in Medians&-1.8&1.9&-1.5 \tabularnewline
95\% confidence interval&.&.&. \tabularnewline
Median Percent Change (p10)*&-52.8&-52.3&-73.8 \tabularnewline
Median Percent Change (p25)*&-29.9&-23.9&-38.8 \tabularnewline
Median Percent Change (p50)&0.1&3.2&-1.5 \tabularnewline
Median Percent Change (p75)*&52.5&42.4&48.9 \tabularnewline
Median Percent Change (p90)*&185.5&111.0&185.5 \tabularnewline
95\% confidence interval (p50)&.&.&. \tabularnewline
\bottomrule \addlinespace[1.5ex]

\end{tabularx}
\begin{flushleft}
\footnotesize *These values are not medians but percentiles, as indicated in the parentheses. \linebreak --- \linebreak This table references Table 1 of Hurd and Rohwedder's paper: Heterogeneity in spending change at retirement. Hurd and Rohwedder bootstrap their confidence intervals. \linebreak --- \linebreak Mean percent change is not reported because observation error on spending can produce large outliers when spending is put in ratio form. \linebreak --- \linebreak Retirement sample, N = 443. This sample consists of households where we have panel data on actual spending pre- and post-retirement, and on the anticipations of spending change prior to retirement and recollections of spending change after retirement. The sample describes retirement transitions among 50 to 70 year-olds where the responses to the question “Are you retired?” indicate a transition from not retired to retired. These responses are constructed from four waves of CAMS, 2001 to 2007, yielding three panel transitions where we observe actual spending data before and after retirement for these observations.
\end{flushleft}
\end{table}
